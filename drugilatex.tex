\documentclass{article}
\usepackage{polski}
\usepackage[latin2]{inputenc}
\usepackage{amsmath, amssymb, amsthm}

\title{Co wiemy o wielomianach}
\author{Tomasz Tylec}
\date{wczoraj}

\begin{document}

\maketitle

Wielomiany s� s�odkie.
Wielomian $a x^{3} + \sqrt{b}$ jest fajny.
\begin{equation}
	\int_{a}^{\infty} \frac{\sin x}{\tg x} dx
\end{equation}
nie jest wielomianem.
\begin{equation}
a\!bc\,g\;d\quad e\qquad f
\end{equation}
$W(x)=x^3 + b x + c$, wtedy 
$W'(x)=3x^2 + b$.
W j�zyku polskim cudzys��w wygl�da ,,tak''.
\begin{equation}
  \vec{r}_0 = \sum_{i=1}^{N} \vec{r}_i
\end{equation}
Wyznacznik macierzy \&:
\begin{equation} 
\left|
\begin{matrix}
 a & b \\
 c & d
\end{matrix}
\right| = 
\det \left[
\begin{matrix}
 a & b \\
 c & d
\end{matrix} 
\right] =
ad - bc
\end{equation}
$\alpha, \beta, \gamma, \Gamma, \Delta, \aleph, \triangle$.

Tr�jk�t nazywamy \emph{prostok�tnym,} gdy ma co najmniej jeden k�t prosty. $\mathbb R, \mathbb C, \mathbb 1$.
\end{document}